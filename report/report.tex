\documentclass[10pt,twocolumn,letterpaper]{article}

% \usepackage[review]{cvpr}      % To produce the REVIEW version
% \usepackage{cvpr}              % To produce the CAMERA-READY version
\usepackage[pagenumbers]{cvpr} % To force page numbers, e.g. for an arXiv version

% Include other packages here, before hyperref.
\usepackage{graphicx}
\usepackage{amsmath}
\usepackage{amssymb}
\usepackage{booktabs}


% It is strongly recommended to use hyperref, especially for the review version.
% hyperref with option pagebackref eases the reviewers' job.
% Please disable hyperref *only* if you encounter grave issues, e.g. with the
% file validation for the camera-ready version.
%
% If you comment hyperref and then uncomment it, you should delete
% ReviewTempalte.aux before re-running LaTeX.
% (Or just hit 'q' on the first LaTeX run, let it finish, and you
%  should be clear).
\usepackage[pagebackref,breaklinks,colorlinks]{hyperref}


% Support for easy cross-referencing
\usepackage[capitalize]{cleveref}
\crefname{section}{Sec.}{Secs.}
\Crefname{section}{Section}{Sections}
\Crefname{table}{Table}{Tables}
\crefname{table}{Tab.}{Tabs.}


%%%%%%%%% PAPER ID  - PLEASE UPDATE
\def\cvprPaperID{*****} % *** Enter the CVPR Paper ID here
\def\confName{CVPR}
\def\confYear{2023}


\begin{document}

\title{Co-Citation Prediction with Graph Networks and Transformers}

\author{Samihan Dani, Gaurav Sett\\
Georgia Insitute of Technology\\
\{sdani30, gauravsett\}@gatech.edu
% For a paper whose authors are all at the same institution,
% omit the following lines up until the closing ``}''.
% Additional authors and addresses can be added with ``\and'',
% just like the second author.
% To save space, use either the email address or home page, not both
% \and
% Second Author\\
% Institution2\\
% First line of institution2 address\\
% {\tt\small secondauthor@i2.org}
}
\maketitle


\begin{abstract}
   A vast number of academic papers are published each year. Especially in fast-paced disciplines like computer science, it is impossible for researchers to develop a comprehensive understanding of the landscape. We aim to build a model to predict the likelihood of two papers being co-cited by subsequent work. This model can help researchers parse through literature to identify novel research directions. We will leverage language models to represent each paper's content and graph convolutions to represent citation networks.
\end{abstract}


\section{Introduction}
\label{sec:intro}

(5 points) What did you try to do? What problem did you try to solve? Articulate your objectives
using absolutely no jargon.
(5 points) Who cares? If you are successful, what difference will it make?
(5 points) What data did you use? Provide details about your data, specifically choose the most
important aspects of your data mentioned here: Datasheets for Datasets
(https://arxiv.org/abs/1803.09010). Note that you do not have to choose all of them, just the most
relevant.


\section{Related Work}
(5 points) How is it done today, and what are the limits of current practice?


\section{Data}
\cite{tang2008arnetminer}

\section{Methods}

(10 points) What did you do exactly? How did you solve the problem? Why did you think it would
be successful? Is anything new in your approach?
(5 points) What problems did you anticipate? What problems did you encounter? Did the very first
thing you tried work?

\subsection{Architecture}

\subsubsection{Transformer Encoder}

\subsubsection{Graph Convolutional Network}

\subsubsection{Regression Head}

\subsection{Evaluation}

\section{Results}

(10 points) How did you measure success? What experiments were used? What were the results,
both quantitative and qualitative? Did you succeed? Did you fail? Why? Justify your reasons with
arguments supported by evidence and data. Make sure to mention any code repositories and/or
resources that you used!

\section{Discussion}


%%%%%%%%% REFERENCES
{\small
\bibliographystyle{ieee_fullname}
\bibliography{egbib}
}

\end{document}
